\usepackage{xcolor}
\usepackage{amsmath}
\usepackage{amsfonts}
\usepackage{amsthm}
\usepackage{amssymb}
\usepackage[brazilian]{babel}
\usepackage[utf8]{inputenc}
\usepackage[T1]{fontenc}
\usepackage{graphicx}
\usepackage{subfigure}
\usepackage{enumerate}
\usepackage{times}
\usepackage{setspace}
\usepackage{booktabs}
\usepackage{tikz}
\usetikzlibrary{decorations.pathreplacing, shapes, arrows.meta, positioning}
\usepackage{bm}
\usepackage{multirow}

\DeclareMathOperator{\Var}{Var}
\DeclareMathOperator{\DP}{DP}
\DeclareMathOperator{\DM}{DM}
\DeclareMathOperator{\espe}{E}

% \usepackage[all]{background}

% \backgroundsetup{
% placement=center,
% scale=1.5,
% color=black,
% opacity=0.1,
% angle=0,
% contents={%
%   \includegraphics[width=5cm]{logo-ufba.png}
%   }%
% }

% \usebackgroundtemplate{}  


%PE - Curso - Introducao a Estatistica usando o R Aplicacao em Analise Laboratoriais
\title[\texttt{R} para Ciência de Dados]{\LARGE{\texttt{R} para Ciência de Dados: \\Exploração e Visualização de Dados}}
\institute[IME-UFBA]{\Large{Instituto de Matemática e Estatística\\ Universidade Federal da Bahia}\\ \vspace{1cm} \large{Profa Carolina \& Prof Gilberto}}