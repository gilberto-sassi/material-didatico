\usepackage{xcolor}
\usepackage{amsmath}
\usepackage{amsfonts}
\usepackage{amsthm}
\usepackage{amssymb}
% \usepackage[brazilian]{babel}
% \usepackage[utf8]{inputenc}
% \usepackage[T1]{fontenc}
\usepackage{graphicx}
% \usepackage{subfigure}
\usepackage{enumerate}
% \usepackage{times}
\usepackage{setspace}
\usepackage{booktabs}
\usepackage{tikz}
\usetikzlibrary{decorations.pathreplacing, shapes, arrows.meta, positioning}
\usepackage{bm}
\usepackage{multirow}
% \usepackage[normalem]{ulem}

\usepackage{hyperref}
\definecolor{titulo}{HTML}{003D1F}
\definecolor{cabecalho}{HTML}{E5F0EB}
\hypersetup{
    colorlinks=true,
    linkcolor=titulo,
    filecolor=magenta,      
    urlcolor=titulo,
    pdftitle={Rmarkdown e Quarto},
    pdfpagemode=FullScreen,
    }

\DeclareMathOperator{\Var}{Var}
\DeclareMathOperator{\DP}{DP}
\DeclareMathOperator{\DM}{DM}
\DeclareMathOperator{\espe}{E}



% \usepackage[all]{background}

% \backgroundsetup{
% placement=center,
% scale=1.5,
% color=black,
% opacity=0.1,
% angle=0,
% contents={%
%   \includegraphics[width=5cm]{logo-ufba.png}
%   }%
% }

% \usebackgroundtemplate{}  

\newcommand*{\destaque}[1]{%
    \colorbox{cabecalho}{\textcolor{titulo}{#1}}
}

\newcommand*{\regrafina}{\rule{\textwidth}{0.5pt}}
\newcommand*{\regragrossa}{\rule{\textwidth}{1pt}}

%PE - Curso - Introducao a Estatistica usando o R Aplicacao em Analise Laboratoriais
% \title[\texttt{R} para Ciência de Dados]{\LARGE{\texttt{R} para Ciência de Dados: \\\texttt{rmarkdown} e \texttt{quarto}}}
% \institute[IME-UFBA]{\Large{Instituto de Matemática e Estatística\\ Universidade Federal da Bahia}\\ \vspace{1cm} \large{Profa Carolina \& Prof Gilberto}}