\documentclass[9pt]{beamer}

%%pacotes referentes ao Beamer
%\useoutertheme{split}
%\setbeamertemplate{navigation symbols}{}

%\usetheme{beamertheme}
%\usebeamercolor{beamer-color name}

%\usetheme{Boadilla}
%\usecolortheme{dove}

\usetheme{CambridgeUS}
\usecolortheme{beaver}

% \usetheme{pittsburgh}
% \usecolortheme{dolphin}
%\usecolortheme{dove}
%\usecolortheme{seahorse}

%\usetheme{Montpellier}

%number in figures an tables -- beamer
\setbeamertemplate{caption}[numbered]

%Colocar no description
%[leftmargin=!,labelwidth=\widthof{Turma A}]
	
%pacotes usuais do latex
\usepackage[portuguese]{babel}
\usepackage[utf8]{inputenc}
\usepackage{bm}
\usepackage{graphicx}
\usepackage{subfigure}
\usepackage[round]{natbib}
\usepackage{tikz}
\usetikzlibrary{shapes,arrows}
\usepackage{natbib}
\usepackage{times}
\usepackage{calc} %computes the length of a string
\usepackage{dsfont} %pacote para o 1 estilisado para indicadora
\usepackage{enumerate} %permite fazer uns enumerates diferentes
\usepackage[font=small,labelfont=bf]{caption} %permite colocar um segundo caption
\usepackage{booktabs} % comando \toprule, \midrule e \bottomrule
\usepackage{times} %times new roman font
\usepackage{multirow} %comando \multirow
\usepackage{setspace}
\usepackage{xcolor} %texto colorido
\usepackage{booktabs} %costumized tabs
\usepackage{color}


%código para alinhar a esquerda os itens no description
\defbeamertemplate{description item}{align left}{\insertdescriptionitem\hfill}
\defbeamertemplate{enumerate item}{align left}{\insertdescriptionitem\hfill}



%AMS packages
\usepackage{amsmath}
\usepackage{amsfonts}
\usepackage{amssymb}

%Não quebre linhas
\binoppenalty=\maxdimen
\relpenalty=\maxdimen

%Comandos criados por mim
\DeclareMathOperator*{\argmin}{arg\,min}

\DeclareMathOperator*{\argmax}{arg\,max}


\DeclareMathOperator{\espe}{E}

\DeclareMathOperator{\spann}{span}

\DeclareMathOperator{\cov}{Cov}

\DeclareMathOperator{\vari}{Var}

%Informações para o primeiro slide
\date{}
\title[Probabilidade]{Conceitos iniciais de probabilidade}
\author[Gilberto Sassi]{Gilberto Pereira Sassi}
\institute[IME -- UFBA]{Universidade Federal da Bahia \\ Instituto de Matem\'{a}tica e Estat\'{i}stica\\ Departamento de Estat\'{i}stica }

\begin{document}
	
\tikzstyle{decision} = [diamond, draw, fill=blue!20, 
text width=4.5em, text badly centered, node distance=3cm, inner sep=0pt]
\tikzstyle{block} = [rectangle, draw, fill=blue!20, 
text width=5em, text centered, rounded corners, minimum height=4em]
\tikzstyle{line} = [draw, -latex]
\tikzstyle{cloud} = [draw, ellipse,fill=red!20, node distance=3cm,
minimum height=2em]
	
\begin{frame}{}
	\maketitle
\end{frame}

\section{Introdução: fenômeno aleatório, espaço amostral e evento.}
%Slide 1
\begin{frame}{}
 \begin{block}{Objetivo}
  Apresentar a teoria matemática para avaliar/estimar/decidir usando as Informações disponíveis.
 \end{block}

 \begin{block}{Definição}
  \begin{enumerate}
   \item \textbf{Fenômeno aleatório}: situações ou acontecimentos que não podem ser previstos com certeza. Por exemplo: condições climáticas em dois dias;
   \item \textbf{Espaço amostral}: conjunto de todos os resultados possíveis de um fenômeno aleatório. 
   
   Por exemplo: $\Omega = \{\mbox{cara}, \mbox{coroa}\}$ no lançamento de uma moeda;
   \item Os elementos de $\Omega$ são denominados de \textbf{pontos amostrais} e usamos a letra grega $\omega$ para representá-lo;
   \item \textbf{Eventos}: subconjuntos de $\Omega$. Representamos eventos por letras do alfabeto latino em maiúsculas.
   
   Por exemplo: Em um lançamento de dado, o espaço amostral é $\Omega = \{1,2,3,4,5,6\}$ e podemos considerar o evento $A = \{\mbox{A face é par}\}$.
  \end{enumerate}
 \end{block}

\end{frame}

%Slide 2
\begin{frame}{Eventos}
% \begin{block}{Definição}
% 	\textbf{Evento}: Subconjunto do espaço amostral $\Omega$. Representamos eventos por letras do alfabeto latino em maiúsculas.
% \end{block}

\begin{block}{Operação com eventos}
	\begin{enumerate}[i.]
		\item \textbf{União}: $A \cup B = \{\omega \in \Omega \mid \omega \in A \mbox{ ou } \omega \in B \}$;
		\item \textbf{Intersecção:} $A \cap B = \{ \omega \in \Omega \mid \omega \in A \mbox{ e } \omega \in B \}$;
		\item \textbf{Complementação:} $A^c =\{ \omega \in \Omega \mid \omega \not\in A \}$;
		\item Se $A \cap B = \emptyset$, então $A$ e $B$ são disjuntos;
		\item Se $A \cap B = \emptyset$ e $A \cup B = \Omega$, então $A$ e $B$ são complementares.
	\end{enumerate}
\end{block}

\begin{block}{Probabilidade de eventos}
	O objetivo  da teoria de probabilidade é atribuir um valor entre 0 e 1 que corresponde a chance do evento $A$ ocorrer. Este valor é chamado de probabilidade e é denotado por $P(A)$.
\end{block}
\end{frame}

%Slide 3
\begin{frame}{Probabilidade}

	{
	\begin{block}{Definição}
		Uma função $P(\cdot)$ é denominada de probabilidade se satisfaz as seguintes condições:
		\begin{enumerate}[i.]
			\item $0 \leq P(A) \leq 1$ para todos os eventos $A \subset \Omega$;
			\item $P(\Omega)=1$ e $P(\emptyset)=0$;
			\item $P\left(A_1 \cup A_2 \cup \dots \cup A_n \right) = P(A_1) + P(A_2) + \dots + P(A_n)$ se $A_i$ e $A_j$ são disjuntos. 
		\end{enumerate}
		\textbf{Observação: } Note que $n$ em {\color{blue} iii.} pode ser infinito.
	\end{block}
	
	\begin{block}{Observação}
	 Note que $A\cup A^c = \Omega$ e $A \cap A^c =\emptyset $, então, usando o item {\color{blue} iii.}  da definição de probabilidade, temos que
		\begin{align*}
		P(\Omega) &= 1 = P(A \cup A^c) = P(A) + P(A^c)
		\end{align*}
		e, consequentemente, $P(A^c) = 1 - P(A)$.
	\end{block}

	}
\end{frame}

%slide 4
\section{Maneira de construir probabilidade}
\subsection{Princípio da equiprobabilidade}
\begin{frame}{Princípio da equiprobabilidade}
	\begin{block}{Princípio da equiprobabilidade}
		Quando as características de um fenômeno aleatório sugerem $N$ resultados possíveis, todos com igual probabilidade de ocorrer, a probabilidade de um evento $A$, com $n$ pontos amostrais, é dada por
		\begin{align*}
		P(A) = \dfrac{n}{N}.
		\end{align*}
	\end{block}
	\begin{block}{Exemplo}
		\textbf{Fenômeno aleatório}: Lançamento de dados junto. Então
		\begin{itemize}
			\item \textbf{espaço amostral}: $\Omega = \{ 1, 2, 3, 4, 5, 6 \}$;
			\item \textbf{Evento}: $A = \{\mbox{face par}\} = \{2, 4, 6\}$;
			\item Usando o princípio da equiprobabilidade, temos que $P(A) = \dfrac{3}{6} = 0,5$.
		\end{itemize}
	\end{block}
\end{frame}

%slide 5
\subsection{Probabilidade frequentista}
\begin{frame}{Probabilidade frequentista}

{\scriptsize
	\begin{block}{Probabilidade frequentista}
		Considere um evento $A$ de um fenômeno aleatório e assuma que podemos realizar várias vezes esse fenômeno. Sejam
		\begin{description}
		 \item[$N$] Número de repetições do fenômenos aleatório;
		 \item[$n$] Número de vezes que o evento $A$ foi resultado do fenômeno aleatório.
		\end{description}
		Então, a probabilidade do evento $A$ é $P(A)=\dfrac{n}{N}$.

	\end{block}
	\begin{block}{Exemplo}
		\begin{enumerate}[i.]
			\item \textbf{Fenômeno aleatório:} lançamento de um dado;
			\item Suponha que um indivíduo repetiu esse fenômeno aleatório 100.000 vezes 
			\scriptsize{
			\begin{table}[ht]
				\centering
				\begin{tabular}{l|c|c|c}
					\toprule[0.05cm]
					Face & Frequência & Proporção & Porcentagem \\ 
					\midrule[0.05cm]
					1 & 16665 & 0,1666 & 16,66\% \\ 
					2 & 16622 & 0,1662 & 16,62\% \\ 
					3 & 16835 & 0,1683 & 16,83\% \\ 
					4 & 16545 & 0,1655 & 16,54\% \\ 
					5 & 16631 & 0,1663 & 16,63\% \\ 
					6 & 16702 & 0,1670 & 16,70\% \\ 
					\midrule[0.05cm]
					Total & 100.000 & 1,0000 & 100,00\% \\
					\bottomrule[0.05cm]
				\end{tabular}
			\end{table}
			}
		\item $P(A) = \dfrac{16.622+16.545+16.702}{100.000} = \dfrac{49.869}{100.000} = 0,4987$, em que $A=\{\mbox{face par}\}=\{2,4,6\}$.
		\end{enumerate}
		
	\end{block}
}
\end{frame}

%slide 6
\subsection{Probabilidade subjetiva}
\begin{frame}{Probabilidade subjetiva}
	\begin{block}{Probabilidade subjetiva}
		O pesquisador utiliza sua experiência, seu conhecimento e sua cognição para determinar a probabilidade de um evento ocorrer.
	\end{block}
	\begin{block}{Exemplo}
		Um especialista em conflitos armados pode atribuir um valor entre 0 e 1 para a tensão entre a Irã e os Estados Unidos se escalar até a guerra total.
	\end{block}
	\begin{block}{Exemplo}
		Um médico pode atribuir uma medida entre 0 e 1 para a plausibilidade de um paciente se recuperar completamente.
	\end{block}
\end{frame}

%slide 7
\subsection{Suposição teórica}
\begin{frame}{Suposição teórica}

	\huge{
	\begin{block}{Suposição teórica}
		Supomos um modelo matemático para a probabilidade dos eventos de um fenômeno aleatório com notação matemática $P_\theta(\cdot)$, em que $\theta$ é um valor real inferido usando a amostra, como veremos nas próximas aulas.
	\end{block}
	}
\end{frame}

%slide 8
\section{Propriedades de probabilidade}
\subsection{Regra da adição de probabilidades}
\begin{frame}{}
\begin{block}{Regra da adição de probabilidades}
	$P(A \cup B) = P(A) + P(B) - P(A \cap B) $
\end{block}
\begin{block}{Exemplo}
	Considere os calouros de engenharia divididos em duas turmas:
	{\scriptsize
	\begin{table}[ht]
		\centering
		\begin{tabular}{l|cc|c}
			\toprule[0.05cm]
			\multirow{2}{*}{Sexo} & \multicolumn{2}{|c|}{Turma} & \multirow{2}{*}{Total} \\ \cmidrule{2-3}
			 & A & B &  \\ 
			\midrule[0.05cm]
			F & 21 & 16 & 37 \\ 
			M & 5 & 8 & 13 \\ \midrule[0.05cm] 
			Total & 26 & 24 & 50 \\ 
			\bottomrule[0.05cm]
		\end{tabular}
	\end{table}
	}


	\begin{enumerate}[i.]
		\item \textbf{Fenômeno aleatório}: Selecione ao acaso um calouro;
		\item \textbf{Eventos}: $F=\{\mbox{Calouro do sexo feminino}\}$ e $\{\mbox{Calouro da turma B}\}$
		\item Usando princípio da equiprobabilidade: $P(F)=\dfrac{37}{50}$, $P(B)=\dfrac{24}{50}$ e $P(F \cap B) = \dfrac{16}{50}$;
		\item Usando a regra da adição: $P(F \cup B) = P(B) + P(F)  - P(B \cap F) = \dfrac{37+24-16}{50}= 0,9$.
	\end{enumerate}
\end{block}
\end{frame}

%slide 9
\section{Probabilidade condicional e independência}
\begin{frame}{Probabilidade condicional e independência}

{\small
\begin{block}{Ideia}
	Alguns fenômenos aleatórios podem acontecer ou ser estudados em etapas. A informação do que ocorreu em um determinada etapa pode influenciar nas probabilidades de ocorrência das etapas sucessivas.
\end{block}

\begin{block}{Definição}
	
	\begin{itemize}
		\item Se $P(B)>0$, então $P(A \mid B) = \dfrac{P(A \cap B)}{P(B)}$;
		\item Se $P(B)=0$, então $P(A \mid B) = P(A)$.
	\end{itemize}
\end{block}

\begin{block}{Observação}

Pela definição de probabilidade condicional, temos que 
		$P(A \cap B) = P(A \mid B) P(B) = P(B \mid A) P(A)$.

\end{block}
}
\end{frame}

%slide 10
\begin{frame}{Exemplo - continuação}
	Sabendo que o calouro de engenharia é do sexo feminino, qual a probabilidade dele ser da turma $A$?
	\vfill
	
	\textbf{Resposta:} 
	\begin{itemize}
		\item \textbf{Eventos}: $F = \{\mbox{Calouro do sexo feminino}\}$ e $A=\{\mbox{Calouro da turma A}\}$.
		\item Usando o princípio da equiprobabilidade, temos que $P(A \cap F) = \dfrac{21}{50}$ e $P(F) = \dfrac{37}{50}$.
		\item Usando probabilidade condicional, temos que 
		\begin{align*}
		P(A \mid F) = \dfrac{P(A \cap F)}{P(F)} = \dfrac{\frac{21}{50}}{\frac{37}{50}} = \dfrac{21}{37} = 0,57.
		\end{align*}
	\end{itemize}
\end{frame}

%slide 11
\begin{frame}{Exemplo}


Um restaurante oferece apenas três opções de pratos: salada Caesar,
prato executivo com carne e prato executivo com peixe. O proprietário sabe que 25\% dos clientes preferem salada Caesar, 40\% dos clientes preferem o prato executivo com carne e 60\% dos 
clientes são homens. Qual a probabilidade de um cliente escolher um prato executivo com peixe? Sabendo que entre os clientes que preferem o prato executivo com peixe 56\% são mulheres, 
qual a probabilidade de um homem escolher o prato executivo com peixe?

\end{frame}

%slide 12
\begin{frame}{Exemplo - Resposta}
\begin{itemize}
	\item \textbf{Eventos:} $S = \{\mbox{Cliente prefere salada Caesar}\}$, $C = \{\mbox{Cliente prefere executivo com carne}\}$, $P=\{\mbox{Cliente prefere prato executivo com peixe}\}$, \\
	$F = \{\mbox{Cliente do sexo feminino}\}$,\\
	$M=\{\mbox{Cliente do sexo masculino}\}$;
	\vfill 
	
	\item Usando a propriedade iii. da definição de probabilidade, temos que
	\begin{align*}
	P(\Omega)= 1 = P(S) + P(C) + P(P) = 0,25 + 0,4 + P(P)
	\end{align*}
	e, então, $P(P) = 1 - 0,65 = 0,35$.
	\vfill
	
	\item Usando probabilidade condicional, temos que
	\begin{align*}
	P(P \mid M) = \dfrac{P(P \cap M)}{P(M)} = \dfrac{P(M \mid P)P(P)}{P(M)} = \dfrac{0,44 \cdot 0,35}{0,6} = 0,25.
	\end{align*}
\end{itemize}
\end{frame}

%slide 13
\subsection{Independência de eventos}
\begin{frame}{Independência de eventos}
\begin{block}{Ideia}
	As vezes, a acorrência (ou não) do evento $B$ de um fenômeno aleatório não afeta a ocorrência (ou não) de evento $A$ de um fenômeno aleatório seguinte.
% 	Pode ocorrer que o resultado anterior de um fenômeno aleatório não altere a probabilidade de um resultado de um fenômeno seguinte.
	Quando isso ocorre, dizemos que os eventos são independentes.
\end{block}
	
\begin{block}{Definição}
	Dois eventos são independentes se a informação da ocorrência (ou não) do evento $B$ não altera a probabilidade de $A$, ou seja,
	\begin{align*}
	P(A \mid B) = P(A).
	\end{align*}
\end{block}

\begin{block}{Observação}
	Se $A$ e $B$ são independentes, então
	\begin{align*}
	P(A \mid B) P(B) = \textcolor{red}{P(A) P(B) = P(A \cap B)} 
	\end{align*} 
	e 
	\begin{align*}
	\textcolor{red}{P(B \mid A)} = \dfrac{P(A \cap B)}{P(A)} = \dfrac{P(A) P(B)}{P(A)} = \textcolor{red}{P(B)}.
	\end{align*}
\end{block}

\end{frame}

%slide 14
\begin{frame}{Exemplo}


Uma empresa produz peças em duas máquinas ($I$ e $II$) que podem apresentar desajustes com probabilidade $0,05$ e $0,10$, respectivamente. 
No início do dia de operação, um teste é realizado e, caso a máquina esteja desajustada, ela ficará sem operar nesse dia passando por revisão técnica. 
Suponha que as duas não sofrem interferência uma da outra. Qual a probabilidade de pelo menos uma máquina funcionar? Qual a probabilidade das duas máquinas precisarem de ajuste no mesmo dia?

\end{frame}

%slide 15
\begin{frame}{Exemplo - solução}
	\begin{itemize}
		\item \textbf{Eventos}: $O_1=\{\mbox{Máquina I está desajustada}\}$ e $O_2=\{\mbox{Máquina II está desajustada}\}$;
		\item \textbf{Probabilidade}: $P(O_1) = 0,05$ e $P(O_2) = 0,1$;
		\item O evento pelo menos uma máquina funciona é descrito por $O_1^c \cup O_2^c$, isto é,
		\begin{align*}
		P[O_1^c \cup O_2^c)] &= P(O_1^c ) + P (O_2^c) - P (O_1^2 \cap O_2^c) \\
		&= (1-P(O_1))  + (1-P( O_2)) - P(O_1^c) P(O_2^c) \\
		&=(1-0,05)  + (1-0,1) - (1-0,05) \cdot (1-0,1) = 0,995.
		\end{align*}
		\item Note que
		\begin{align*}
		P(\Omega) &= 1 =  P\left(\left( O_1^c \cup O_2^c \right)\bigcup \left(O_1^c \cup O_2^c\right)^c\right)\\
		&= P\left(\left( O_1^c \cup O_2^c \right)\right) + P\left(\left( O_1 \cap O_2 \right)\right)\\
		\end{align*}
		e, então, $P(O_1 \cap O_2) = 1 - 0,995 = 0,005$.
	\end{itemize}	
\end{frame}

%slide 16
\subsection{Teorema da probabilidade total}

\begin{frame}{Partição}


		Os eventos $C_1, \dots, C_k$ formam uma partição do espaço amostral $\Omega$ se 
		\begin{enumerate}[i.]
			\item $C_i \cap C_j= \emptyset$ se $i \neq j$, ou seja, $C_i$ e $C_j$ são disjuntos;
			\item $C_1 \cup C_1 \cup \dots \cup C_k  =  \Omega$.
		\end{enumerate}
		
	 \begin{figure}[htbp]
	 \caption{Ilustração de uma partição.}
	  \begin{tikzpicture}
	   \draw [thick] (0,0) rectangle (4,4);
	   \draw (0,2) -- (2, 4);
	   \draw (0, 2) -- (2, 0);
	   \draw (2,4) -- (4, 0);
	   \node [above] at (1,3) {{\scriptsize $C_1$}};
	   \node [below] at (1,1) {{\scriptsize $C_2$}};
	   \node [below] at (2,2) {{\scriptsize $C_3$}};
	   \node [right, above] at (3.5, 3) {{\scriptsize $C_4$}};
	   \node [left, above] at (3.8,4.2) { $\Omega$};
	  \end{tikzpicture}
	 \end{figure}
\end{frame}


\begin{frame}{Teorema da probabilidade total}

	Considere $C_1, C_2, \dots, C_k$ uma partição de $\Omega$ e o eventos $A \subset \Omega$, então
		\begin{align*}
		P(A) = P(A \mid C_1) P(C_1) + P(A \mid C_2) P(C_2) + \cdots + P(A \mid C_k) P(C_k).
		\end{align*}


		\begin{figure}[htbp]
		\caption{Ilustração -- Teorema de probabilidade total.}
	  \begin{tikzpicture}
	   \draw [thick] (0,0) rectangle (4,4);
	   \draw[fill=cyan] (2,2) circle [radius=2];
	   \node [thick, above, left] at (2,3) {{\huge \color{blue} $A$}};
	   \draw (0,2) -- (2, 4);
	   \draw (0, 2) -- (2, 0);
	   \draw (2,4) -- (4, 0);
	   \node [above] at (1,3) {{\scriptsize $C_1$}};
	   \node [below] at (1,1) {{\scriptsize $C_2$}};
	   \node [below] at (2,2) {{\scriptsize $C_3$}};
	   \node [right, above] at (3.5, 3) {{\scriptsize $C_4$}};
	   \node [left, above] at (3.8,4.2) { $\Omega$};
	  \end{tikzpicture}
	 \end{figure}

\end{frame}


%slide 17
\begin{frame}{Exemplo}


Suponha que um fabricante de sorvetes recebe 20\% de todo o leite que utiliza fazenda $F_1$, 30\% da fazenda $F_2$ e 50\% da fazenda $F_3$. 
A ANVISA inspecionou as fazendas de surpresa e observou que 20\% do leite produzido por $F_1$ estava adulterado com água, enquanto que $F_2$ e $F_3$ essa era de 5\% e 2\%, respectivamente. 
Na planta industrial da fabricante de sorvetes, os galões de leite são armazenados sem identificação de origem. Para um galão escolhido ao acaso, qual a probabilidade do leite estar adulterado?

\end{frame}

%slide 18
\begin{frame}{Exemplo - solução}
	\begin{itemize}
		\item \textbf{Eventos}: $A=\{\mbox{Galão adulterado}\}$, $F_1 =\{\mbox{Galão da fazenda }F_1\}$, $F_2 =\{\mbox{Galão da fazenda }F_2\}$ e $F_3 =\{\mbox{Galão da fazenda }F_3\}$;
		\item \textbf{Probabilidades}:
		\begin{table}[ht]
			\begin{tabular}{l|l}
				\toprule[0.05cm]
				$P(A \mid F_1) = 0,2$ & $P(F_1) = 0,2$  \\
				$P(A \mid F_2) = 0,05$ & $P(F_1) = 0,3$  \\
				$P(A \mid F_3) = 0,02$ & $P(F_3) = 0,5$  \\
				\bottomrule[0.05cm]
			\end{tabular}
		\end{table}
	\item Usando o teorema da probabilidade total, temos que
	\begin{align*}
	P(A) &= P(A \mid F_1) P(F_1) + P(A \mid F_2) P(F_2) + P(A \mid F_2) P(F_3)\\
	&= 0,2 \cdot 0,2 + 0,05 \cdot  0,3 + 0,02 \cdot 0,5\\
	&= 0,065.
	\end{align*}
	\end{itemize}
\end{frame}

%slide 19
\subsection{Teorema de Bayes}
\begin{frame}{Teorema de Bayes}

{\tiny
	\begin{block}{Ideia}
		Conhecendo as probabilidades $P(A \mid B)$, $P(A)$ e $P(B)$, desejamos calcular a probabilidade $P(B \mid A)$. \textbf{Interpretação}: se $A$ é um sintoma e $B$ é um doença, um médico deseja calcular a probabilidade do paciente ter a doença $B$ se o paciente tem o sintoma $A$, isto é, $P(B \mid A)$.
	\end{block}

	\begin{block}{Teorema de Bayes}
		Considere $C_1, C_2, \dots, C_k$ uma partição do espaço amostral $\Omega$ e seja $A \subset \Omega$ um evento. Assuma que conhecemos as probabilidades $P(A \mid C_1), P(A \mid C_2), \dots, P(A \mid C_k), P(C_1), P(C_2), \dots, P(C_k)$. Então,
		\begin{align*}
		P(C_j \mid A) = \dfrac{P(A \mid C_j)P(C_j)}{P(A \mid C_1)P(C_1)+P(A \mid C_2)P(C_2) + \cdots + P(A \mid C_k)P(C_k)}.
		\end{align*}
		 em que $j =1, \dots, k$.
	\end{block}
	
	\begin{block}{Interpretação}
	 Suponha que $C_1, \dots, C_k$ são defeitos ou falhas que apresentam o mal funcionamento $A$ de um determinado equipamento. 
	 Assuma que  conhecemos as probabilidades do equipamento com o defeito $C_i$ ter o mal funcionamento $A$:$P(A \mid C_1), \dots, P(A \mid C_k)$ 
	 e a probabilidade do equipamento ter o defeito $C_i$: $P(C_1), \dots, P(C_k)$. Então, se o equipamento tem o mal funcionamento $A$, ele tem o defeito $C_i$ com probabilidade
	 \begin{align*}
	  P(C_i \mid A) = \dfrac{P(A \mid C_i)P(C_i)}{P(A \mid C_1)P(C_1)+P(A \mid C_2)P(C_2) + \cdots + P(A \mid C_k)P(C_k)},
	 \end{align*}
	 para $i=1, \dots, k$.
	\end{block}
}
\end{frame}

\begin{frame}{Exemplo}

{\scriptsize
Suponha que um fabricante de sorvetes recebe 20\% de todo o leite que utiliza fazenda $F_1$, 30\% da fazenda $F_2$ e 50\% da fazenda $F_3$. A ANVISA inspecionou as fazendas de surpresa e observou que 20\% do leite produzido por $F_1$ estava adulterado com água, enquanto que $F_2$ e $F_3$ essa porcentagem era de 5\% e 2\%, respectivamente. Na planta industrial da fabricante de sorvetes, os galões de leite são armazenados sem identificação de origem. A equipe do controle de qualidade testou um galão e verificou que ele está adulterado, qual a probabilidade dele ser proveniente da fazenda $F_1$?
\vfill

\textbf{Solução:}
	\begin{itemize}
	\item \textbf{Eventos}: $A=\{\mbox{Galão adulterado}\}$, $F_1 =\{\mbox{Galão da fazenda }F_1\}$, $F_2 =\{\mbox{Galão da fazenda }F_2\}$ e $F_3 =\{\mbox{Galão da fazenda }F_3\}$;
	\item \textbf{Probabilidades}:
	\begin{table}[ht]
		\begin{tabular}{l|l}
			\toprule[0.05cm]
			$P(A \mid F_1) = 0,2$ & $P(F_1) = 0,2$  \\
			$P(A \mid F_2) = 0,05$ & $P(F_1) = 0,3$  \\
			$P(A \mid F_3) = 0,02$ & $P(F_3) = 0,5$  \\
			\bottomrule[0.05cm]
		\end{tabular}
	\end{table}
	\item Usando o Teorema de Bayes, temos que
	\begin{align*}
	P(F_1 \mid A) &= \dfrac{ P(A \mid F_1) P(F_1) } {P(A \mid F_1) P(F_1) + P(A \mid F_2) P(F_2)+ P(A \mid F_3) P(F_3)}\\
	&= \dfrac{0,2 \cdot 0,2}{0,2 \cdot 0,2 + 0,05 \cdot 0,3 + 0,02 \cdot 0,5} = 0,62.
	\end{align*}
	\end{itemize}
}
\end{frame}

% \section{Aplicações do Teorema de Bayes à Saúde}
% 
% 
% \begin{frame}{Alguns Conceitos Básicos}
% 
% {\small
% Um importante problema na área de saúde concerne o resultado de testes de diagnóstico e a probabilidade do paciente estar doente. Considere os eventos $D=\{\mbox{Paciente está doente}\}$ e $P=\{\mbox{O exameu deu positivo para doença.}\}$. Note que o exame pode errar e temos os seguintes conceitos:
% \begin{itemize}
% 	\item \textbf{Valor Preditivo Positivo:} Probabilidade do paciente ter a enfermidade dado que o exame deu positivo -- $P(D \mid P)$;
% 	\item \textbf{Valor Preditivo Negativo:} Probabilidade do paciente não ter a doença dado que o exame deu negativo -- $P(D^c \mid P^c)$;
% 	\item {\bf Sensibilidade: } Probabilidade do exame dar positivo dado que o paciente está doente -- $P(P \mid D)$;
% 	\item {\bf Especificidade: } Probabilidade do exame dar negativo dado que o paciente não está doente -- $P(P^c \mid D^c) $;
% 	\item {\bf Falso Positivo: } O exame deu positivo, mas a pessoa não tem a doença -- $P(P \mid D^c)$;
% 	\item {\bf Falso Negativo: } O exame deu negativo, mas a pessoa está doente -- $P(P^c \mid D)$;
% \end{itemize}
% 
% }
% \end{frame}
% 
% \begin{frame}{Alguns conceitos básicos -- continuação}
% \begin{block}{Observação}
% 	\begin{itemize}
% 		\item A sensibilidade e a especificidade devem ser altos para um exame ser efetivamente preditivo da doença;
% 		\item Podemos substituir a palavra ``exame'' por ``sintoma'';
% 		\item Note que
% 		\begin{align*}
% 		&\mbox{Sensibilidade} + \mbox{Falso negativo} = 1,\\
% 		&\mbox{Especificidade} + \mbox{Falso positivo} = 1;
% 		\end{align*}
% % 		$1-P(\{\mbox{Falso Negativo}\}) = \mbox{Sensibilidade}$, e $1-P(\{\mbox{Falso Positivo}\}) = \mbox{Especificidade}$.
% 	\end{itemize}
% \end{block}
% 
% \begin{block}{Sensibilidade, Especificidade, Falso Positivo e Falso Negativo}
% \begin{figure}[htbp]
%  \centering
%  \begin{tikzpicture}[scale = 0.25]
%   \draw (0,0) -- (3,2); 
%   \draw (0,0) -- (3,-2);
%   \node[right] at (3,2) {$D$};
%   \node[right] at (3,-2) {$D^c$};
%   \draw (5,2) -- (9,3.5);
%   \draw (5,2) -- (9,0.9);
%   \node[right] at (9,3.5) {$P$ -- Sensibilidade};
%   \node[right] at (9,0.9) {$P^c$ -- Falso Negativo};
%   \draw (5,-2) -- (9,-0.9);
%   \draw (5,-2) -- (9,-3.5);
%   \node[right] at (9,-0.9) {$P$ -- Falso Positivo};
%   \node[right] at (9,-3.5) {$P^c$ -- Especificidade};
%  \end{tikzpicture}
% \end{figure}
% 
%  
% \end{block}
% 
%  
% \end{frame}
% 
% 
% %slide 20
% \begin{frame}{Exemplo}
% 
% {\small 
% A prevalência do vírus HIV no Brasil é estimada em  1\% da sua população. Suponha que o governo federal desenvolveu um novo exame que tem probabilidade 1\% de \textit{Falso Negativo} e probabilidade 2\% de \textit{Falso Positivo} para ser usado no SUS. Calcule a especificidade, sensibilidade, valor preditivo negativo e o valor preditivo positivo.
% \vfill
% 
% \textbf{Solução:} Do enunciado, sabemos que $P(D)=0,01$, $P(P \mid D^c) = 0,02$ e $P(P^c \mid D) = 0,01$, em que $D=\{\mbox{Paciente está infectado com HIV}\}$ e $P=\{\mbox{O exame deu positivo para HIV}\}$.
% \begin{itemize}
%  \item \textbf{Especificidade}: $P(P^c \mid D^c) = 1 - P(P \mid D^c) = 1-0,02=0,98$;
%  \item \textbf{Sensibilidade}: $P(P \mid D) = 1 - P(P^c \mid D) = 1-0,01=0,99$;
%  \item \textbf{Valor Preditivo Negativo: } Desejamos calcular $P(D^c \mid P^c)$. Usando o teorema de Bayes, temos que
%  \begin{align*}
%   P(D^c \mid P^c) &= \dfrac{P(P^c \mid D^c)P(D^c)}{P(P^c \mid D^c)P(D^c) +P(P^c \mid D)P(D)}\\
%   &= \dfrac{0,98 \cdot (1-0,0,1)}{0,98 \cdot (1-0,01) + 0,01 \cdot 0,01} = 0,98
%  \end{align*} 
%  \item \textbf{Valor Preditivo Positivo: } Desejamos calcular $P(D \mid P)$. Usando o teorema de Bayes, temos que
%  \begin{align*}
%   P(D \mid P) &= \dfrac{P(P \mid D)P(D)}{P(P \mid D)P(D)+ P(P \mid D^c)P(D^c)}\\
%   &= \dfrac{0,99 \cdot 0,01}{0,99 \cdot 0,01 + 0,02 \cdot (1-0,01)} = 0,33
%  \end{align*}
% 
% \end{itemize} 
% 
% }
% \end{frame}
% % 
% % \section{Exercícios em sala de aula}
% % %slide 21
% % \begin{frame}{Exercícios}
% % 	\begin{enumerate}[1)]
% % 	\item Das pacientes de uma Clínica de Ginecologia com idade acima de 40 anos, $60\%$ são ou foram casadas e $40\%$ são solteiras. Sendo solteira, a probabilidade de ter distúrbio hormonal no último ano é de $10\%$, enquanto que para as demais essa probabilidade aumenta para $30\%$. Pergunta-se:
% % 	\begin{enumerate}[a)]
% % 	 \item Qual a probabilidade de uma paciente escolhida ao acaso ter tido um distúrbio hormonal?
% % 	 \item Se a paciente sorteada tiver distúrbio hormonal, qual a probabilidade de ser solteira?
% % 	 \item Se escolhermos duas pacientes ao acaso e com reposição, qual é a probabilidade de pelo menos uma ter o distúrbio?
% % 	\end{enumerate}
% % 	\item Numa certa população, a probabilidade de gostar de teatro é $\frac{1}{3}$, enquanto que a de gostar de cinema é $\frac{1}{2}$. Determine a probabilidade de gostar de teatro e não de cinema, nos seguintes casos:
% % 	\begin{enumerate}[a)]
% % 	 \item Gostar de teatro e de cinema são eventos disjuntos;
% % 	 \item Gostar de teatro e de cinema são eventos independentes;
% % 	 \item Todos que gostam de teatro gostam de cinema;
% % 	 \item A probabilidade de gostar de teatro e cinema é $\frac{1}{8}$.
% % 	\end{enumerate}	
% % 	\item Um médico desconfia que um paciente tem tumor no abdômen, pois isto ocorreu em $70\%$ dos casos similares que tratou. Se o paciente de fato tiver o tumor, o exame ultra-som detectará com probabilidade de 0,9. Entretanto, se ele não tiver o tumor, o exame pode, erroneamente, indicar que tem com probabilidade 0,1. Se o exame detectou um tumor, qual é a probabilidade do paciente tê-lo de fato?
% % 
% % 	\end{enumerate}
% % \end{frame}
\end{document}